\documentclass[a4paper]{article}
\usepackage[utf8]{inputenc}
\usepackage{geometry}
\usepackage{fancyhdr}
\usepackage[ngerman]{babel}
\usepackage{amsmath,amssymb,amsthm}
\usepackage{blindtext}


\begin{document}
%Globale Einstellungen
\renewcommand{\rmdefault}{\sfdefault}
\newcommand{\quoderat}{\hfill$\blacksquare$}
\setlength{\parindent}{0pt}
\pagestyle{fancy}
%Header
\lhead
  {
  Vorlesung}
\chead
  {
  \Large Abgabe 1}
\rhead
  {
  WiSe 20/21}
%Body
\begin{flushright}
\textsf{
\begin{tabular}{cc}
Özgül Schneider & (654321) \\
Yvonne Nguyen-Chang & (612345)
\end{tabular}}
\end{flushright}
\paragraph{Aufgabe 1} Da dies ein Test ist, zeigen wir im Folgenden, was zu zeigen ist. Sei hierfür $\varepsilon < 0$.
Offensichtlich ist dann $-\varepsilon > 0$, was zu zeigen war.\quoderat
\paragraph{Aufgabe 2} \blindtext\quoderat
\paragraph{Aufgabe 3} \blindtext
Also gilt
\[\int\limits_\Lambda \Gamma(\delta_i)=\Theta^i_{\delta_i}.\]
\blindtext
Der Rest ist trivial und bleibt dem geneigten Leser überlassen.\quoderat
%Footer
\cfoot
  {
  \hphantom{X}}
\rfoot
  {\thepage}
\end{document}
